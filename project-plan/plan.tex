\title{Declarative and Reactive Serverless Web Applications (Final Project Plan)}
\author{
  Team \#2: Donald Pinckney
}
\date{\today}

\documentclass[12pt]{article}

\usepackage{amsmath}
\usepackage{amsthm}
\usepackage{graphicx}

\begin{document}
\maketitle

To the reader: apologies for the somewhat verbose
project plan. You can safely skip Section 1 if needed.

\section{Overview}

\subsection{Introduction}

Serverless functions are a recently introduced cloud
computing abstraction which allows developers to
upload functions to a serverless platform (e.g., AWS Lambda),
and later invoke them on-demand.
The platform handles all work of dynamically allocating server resources 
for running the function. A potential application is writing 
\emph{serverless Web applications}, in which progrmmers implement
backend logic code in serverless functions, and frontend code invokes these
serverless functions as needed over HTTP. Such an architecture can free
programmers from managing server resources, increasing availability, and lowering
costs.

Unfortunately, programming solely using serverless 
functions is currently difficult, because function 
composition is not well-supported for vanilla serverless functions.
Baldini et al.~\cite{baldini:trilemma} discuss that unless
serverless platforms provide special support for composition,
composition of functions is inherently flawed. 
The main platform that supports serverless function composition
via additional primitives is AWS Lambda Step Functions which
which allows programmers to specify a sequence of 
serverless functions. Unfortunately, the specification format is
an ad-hoc language nearly unreadable to programmers, 
and it is unclear if it provides sufficiently flexibility.

\subsection{An Alternative Approach}

Rather than serverless platforms treating serverless functions
alone as the primitive, and composition as an
afterthought, I propose that serverless platforms
should allow programmers to deploy function composition pipelines
as one unit, and the serverless platform does 
the work of managing the individual component functions.

Second, by allowing the client and serverless platform to have rich
information about the compositional structure, the client and
server can decide where each function should be placed 
(e.g. in geographic server region or on the client itself)
and with what CPU and RAM resources
so as to minimize total latency and financial cost.

\textbf{Specifically, the contributions of this work will be:}
\begin{itemize}
  \item A programming model which is convenient and expressive
    for programmers to write serverless Web applications in,
    and allows them to make easy use of serverless function
    composition.
  \item A new serverless platform which allows both functions and 
    static composition structure data to be deployed in the format
    from the above programming model.
  \item An optimization algorithm which allows the serverless
    platform to optimize deployment and provisioning of
    serverless function resources to minimize latency 
    and cost.
\end{itemize}

\subsection{Background and Related Work}

\paragraph{Arrows and Functional Reactive Programming}
A natural programming model for writing restricted forms of
functional composition is Hughes's 
Arrow model~\cite{hughes:arrows}, and has been used
for writing interactive applications such as in 
the work on 
Arrowized Functional Reactive Programming~\cite{nilsson:afrp}.
In a related vein, Functional Reactive Programming has also
been used as an abstraction for writing interactive Web
applications (client-side only)~\cite{flapjax}.
A highly limited version of using Hughes's Arrows for
serverless composition was explored in my
previous OOPSLA 2019 paper on 
serverless computing~\cite{jangda:lambda-lambda}.

In this project I aim to extend the compositions to cross the
client-cloud boundary so that programmers don't need to separately
manage their serverless functions from their frontend code.
I believe that my programming model will make use of,
or at least be heavily influence by, Arrowized Functional
Reactive Programming.

\paragraph{Automatic Partitioning of Web Applications}
Prior work has explored automatic placement of code onto
either the client or server~\cite{chong:swift}, to minimize
a given cost function.
In many ways I am trying to explore a version of this line of
work but for a serverless world, which requires a more
restrictive programming model.

\section{Goals}

\subsection{System Goals}

While I have lots of thoughts for this work, for the purposes
of a one month long final project I want to keep my goals
attainable. My specific goals are:

\begin{enumerate}
  \item Design a (prototype) programming model for writing
    function pipelines that can then be deployed to my
    serverless platform. This programming model
    will be an embedded DSL in Javascript, and will likely
    be a version of Hughes's Arrows / Arrowized FRP.
    As a bare minimum I will support function composition
    in this DSL, but if time allows I would try to add
    more features such as persistent storage and event
    reactivity.
  \item Implement a (prototype) serverless platform
    which will take as input a program written in the embedded DSL,
    extract out the individual functions, and setup the functions
    to run in sequence to implement to correct semantics
    of the function composition. My project will focus on getting
    the semantics correct, rather than on handling high throughput,
    so for this (first) prototype I will \emph{not} implement
    a serverless platform that runs the individual serverless
    functions on separate compute node, rather all will run
    on a single compute node, though likely in separate threads.
    I will run this on a single Khoury VDI machine.
  \item Develop an optimization algorithm that can let the
    serverless platform decide if a single serverless function
    should be run in the cloud, or if the client itself should
    actually run it. Namely, functions should be placed in the
    cloud / client so as to minimize bytes transfered over the
    Internet. 
    Following the work in~\cite{chong:swift} this
    may be able to be cast as an instance of a graph 
    min-cut problem.
    I plan to explore and develop this algorithm conceptually
    for my final project, but I am unsure that I will have
    the time to implement it fully as part of the serverless
    platform.
\end{enumerate}




\section{System Design}

\subsection{Assumptions}

For this project, I want to focus on developing
a) the programming model, and b) developing the corresponding
serverless platform with correct semantics. Thus, I will assume:
\begin{enumerate}
  \item Failures: I assume that failures of the serverless platform
  and / or network errors may occurr. In these cases the client
  code should react gracefully, and ideally resume working when
  connectivity is restored. That being said, I am unsure how much
  time I will have to get to implementing failure cases,
  but I would at least like to keep them in mind in terms of
  design.
  \item Workload: I assume that there will be relatively light
  workload on the serverless platform, since I want to focus
  on a simpler and more correct implementation. However, I
  will allow for concurrent execution of serverless functions,
  since that is a really important aspect of semantics in practice.
  \item Users: I assume that users are non-malicious with regards
  to serverless functions they deploy on the platform.
  In particular I don't want to spend time working on execution
  isolation, since this would require lots of engineering work
  and isn't very relevant to the semantics.
  \item Supported Programs: To keep the project simple, I will
  only support small subset of a potentially more advanced programming model.
  In particular, I will only support pure functions without
  side effects or persistent state.
\end{enumerate}

\subsection{Modules of the System}

\subsubsection{Programming Model DSL}
\begin{itemize}
  \item The programming model for writing 
  serverless function compositions will be a DSL embedded
  in Javascript. 
  \item Serverless function compositions will be
  defined in the client-side Javascript code, and then
  deployed as-needed to the serverless platform.
  \item At a bare minimum the DSL will include the primitives
  \texttt{arr} (lift a JS function to a serverless function) 
  and \texttt{f >>> g} (compose f followed by g). Depending on
  how exactly I handle function arguments there may be additional
  primitives, but that will be worked out later.
  \item A serverless function composition (e.g. \texttt{f >>> g >>> h})
  will also have a serialization / deserialization implementation,
  so that these compositions can be sent over the network to the 
  serverless platform.
\end{itemize}

\subsubsection{Clients (Web application Frontends)}
\begin{itemize}
  \item The front-end code of web applications will aim
  to implement the bulk of their functionality inside
  serverless function compositions defined in the DSL outlined
  above. The front-end code will be able to attach \emph{local}
  event listeners to know when the final step in a composition
  has executed, and can then appropriately updated the DOM, etc.
  \item The client will be responsible for deploying
  the serverless function composition to the serverless platform 
  orchestrator.
  This will be automatically managed by a runtime library in the
  client so it doesn't get in the way of the Web application logic.
  \item The client will then communicate with the serverless
  platform orchestrator over the Internet to invoke a serverless function
  composition, and receive response notifications back.
\end{itemize}

\subsubsection{Serverless Platform Orchestrator}
\begin{itemize}
  \item The serverless platform orchestrator manages
  a thread pool of serverless platform workers.
  \item The serverless platform orchestrator will wait for clients to connect
  and upload serverless function compositions.
  \item Upon receiving a serverless function composition from a
  client, the serverless platform orchestrator 
  will setup the necessary data structures to ensure 
  correct execution order of the functions in the composition.
  \item Upon receiving a request from a client to invoke
  a serverless platform composition, the serverless platform
  orchestrator will send the task for the first function in the composition
  to one of the workers in the thread pool.
  \item When a worker finishes executing a function,
  the orchestrator will update its data structures
  and then schedule the next function in the composition
  to run appropriately.
  \item \emph{Optional: the serverless platform orchestrator
  may instead decide to ask the client to execute a serverless
  function task rather than a worker in the cloud, so as to
  allow for minimize bytes transfered over the network.
  The decision would be based on the analysis from the Orchestration
  Optimizer (see below).
  This is a stretch goal and might not be implemented!} 
  \item When the final function in a composition completes,
  the orchestrator will send the result to the client.
\end{itemize}

\subsubsection{Serverless Platform Worker}
\begin{itemize}
  \item A serverless platform worker thread waits for the
  orchestrator to assign it a task to execute, which consists of
  the function definition to run, and input arguments.
  \item A worker then executes the function, and when finished
  notifies the orchestrator of the result.
\end{itemize}

\subsubsection{Orchestration Optimizer (stretch goal, might not be implemented)}
\begin{itemize}
  \item The orchestration optimizer will routinely ask the serverless
  platform orchestrator to collect dynamic trace information
  (e.g. \# of bytes sent between functions).
  \item The orchestration optimizer will use this data to solve a
  minimization problem to decide whether each serverless function
  should be run in the cloud, or on the client. This minimization
  problem would be solved via a MAX-SMT encoding with a solver
  such as Z3.
  \item The orchestration optimizer will notify the serverless
  platform orchestrator of the new, optimal placements.
\end{itemize}


\section{Evaluation}

The evaluation will focus on correctness rather than performance.
For all evaluation, the serverless platform would be
hosted on a VDI machine.

\subsection{Serverless Function Composition Unit Tests}

The following serverless function compositions should be
a) expressible in the Javascript DSL, and
b) should be able to be deployed to the serverless platform 
and invoked, with correct behavior.
\begin{enumerate}
  \item Execute functions A, B, C, and D in sequence, passing
  the return value of one to the parameter of the next (sequential DAG).
  \item Execute function A, pass its return value to B and C,
  execute B and C in parallel, then execute D passing
  the return values of B and C (diamond DAG).
\end{enumerate}

In addition, the above serverless function compositions should
be able to handle concurrent invokations.

For running these serverless function composition
unit tests, they could be deployed and invoked from any
either from a browser or via command line, on any machine
as long as it has connectivity to VDI.

\subsection{Stretch goal: Unit Test of Optimization}

\emph{If the orchestration optimizer is also completed,}
then the following function composition will be used to test it:
\[
  A \texttt{ >>> } B \texttt{ >>> } C
\]
where:
\begin{itemize}
  \item $A$ takes as input only a few bytes
  \item $A$ returns a value which is also only a few bytes
  \item $A$ is constrained to run on the cloud (e.g. by a programmer-preference in the DSL)
  \item $B$ takes as input the output of $A$ (a few bytes)
  \item $B$ then returns a value which is a large number of bytes (e.g. a couple of megabytes), perhaps an image, video, etc.
  \item $B$ is able to run on either the cloud or the client
  \item $C$ takes as input the output of $B$ (a large amount of data)
  \item $C$ is constrained to run on the client (e.g. so it can access the DOM)
\end{itemize}
By default, the orchestrator will assume that $B$ should be run on the cloud.
However, this is non-optimal since a large amount of data is being transferred over the network.
The orchestration optimizer should be able to place $B$ on the client instead.

\subsection{Building Web Applications Using the Serverless Platform}

To check how viable the programming model is for building
actual Web applications, I will develop a small but
fully-functional Web application which utilizes the serverless
platform to perform some of the computation in the cloud.

An example Web application I expect I could build is:
a Web application in which the user can type in a URL,
and the Web app will fetch data from that URL, analyze it
and see what the top 20 most frequent words are. The fetching
and analysis should be able to be run in the cloud via
my serverless platform.

For running the Web application, the Web application HTML / JS
can loaded simply from a local file on any machine that has
connectivity to VDI.

\section{Project Timeline}

\begin{enumerate}
  \item Develop the Javascript DSL in which to express \emph{basic}
  serverless function compositions. At this point, unit tests 
  \#1 and \#2 should be expressible, as well as any arbitrary
  static DAG. However, these function composition programs
  will not yet be runnable. Estimated completion date: Nov 16.
  \item Implement the serverless platform. At this point, the
  unit tests \#1 and \#2 should be able to be run by deploying
  them to the serverless platform, and invoking them. Estimated completion date: Nov 21.
  \item Debugging time! Estimated completion date: Nov 24.
  \item Implement the example Web application. Estimated completion date: Nov 26.
  \item Nov 26: Start writing the final report \& planning video presentation. Should take a few days.
  \item Flex time to handle project taking longer than expected, or time
  to implement the orchestration optimizer.
\end{enumerate}

% Project title, project team members, and project team # (according to canvas)
% Overview of what you plan to do
% Clear goals
% Detailed design of your system (this can change)
%     Assumptions that you make (e.g., failure, workload, users, etc.)
%     Modules you plan to create
% How you would evaluate your system
%     Every basic/key functionality should have an attached evaluation plan
%     Evaluation should clearly show that the design works
%     Evaluation plan should include how you would run the software in the cloud
% Step by step plans to build/evaluate your system with a timeline


% \bibliographystyle{ACM-Reference-Format}
\bibliographystyle{plain}
\bibliography{main}


\end{document}